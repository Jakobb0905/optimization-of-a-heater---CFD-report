% !TEX TS-program = pdflatex
% !TEX encoding = UTF-8 Unicode
% !TEX root = ../main.tex
% !TEX spellcheck = en-US
% ****************************************************************************************
% File: measurements.tex
% Author: Patrick Haselwanter
% Date: 2023-10-28
% ****************************************************************************************
\chapter{Hardware Implementation and Results}
\label{chapter:measurements}
\section{Hardware Implentation}
\label{sec:HI}
The setup of the real hardware is shown in \cref{fig:measurement_test_setup}.
\begin{figure}[htbp]
	\centering
	\includegraphics[width=0.5\textwidth]{img/setup.jpg}
	\caption{Real hardware setup}
	\label{fig:measurement_test_setup}
\end{figure}
It is composed of a cylinder in which a table tennis ball can move up and down. A fan mounted under the cylinder serves as the actuator that moves the table tennis ball. The position of the ball is read by a sensor mounted above the cylinder. 
The fan is controlled via a separate power supply module. This converts the input voltage of 5V into a corresponding PWM signal depending on the control setpoint. The control setpoint is defined via a controller implemented in Simulink, Simulink communicates with an I/O module from National Instruments (NI) and this passes the analog output to the power supply module. 
There is a similar setup for reading out the sensor, here the analog signal is forwarded to Simulink via the I/O module from NI. 

To apply the controller described in chapter \ref{chap:contr_des} to a real system, the Simulink-file shown in chapter \ref{chap:sim} is used as a basic setup. Instead of the identified system of the real plant, some interfaces to the real plant have to be introduced. This is done with the \textit{Analog Input} block, which builds the interface to the distance sensor and an \textit{Analog Output} block, which builds the interface to the fan motor. For the hardware of the laboratory at MCI VI, the drivers for the I/O Module \textit{National Instruments PCI 6221} have to be installed on the PC. 

To be able to do the system identification described in chapter \ref{sec:sys_ident}, an offset and a gain had to be applied to the response of the system. In order to be able to use the same controller as for the identified system, this offset of $d=0.77$ and gain of $k=1.81$ has to be introduced now to the real system. In the Simulink model, this has been done with simple \textit{gain} and \textit{bias} blocks. 

The following figure shows the Simulink system with the introduced gain and offset-blocks. Additionally, this figure shows the analog in- and output blocks, needed as an interface to the real hardware. 
\begin{figure}[!h]
\centering
    \begin{subfigure}{\textwidth}
      \centering
      \includegraphics[width=1\linewidth]{img/real_layer1.png}
      \caption[Top level servo controlled system]{Top level servo controlled system with interface to the real hardware and introduced offsets ad gains}
      \label{fig:simlayer1}
    \end{subfigure}
    \begin{subfigure}{\textwidth}
      \centering
      \includegraphics[width=.8\linewidth]{img/real_layer2.png}
      \caption{Linear system with observer}
      \label{fig:simlayer2}
    \end{subfigure}   
    \caption[Simulink model used for the real hardware implementation]{Simulink model used for the real hardware implementation}
    \label{fig:Simulink_sim_model}
\end{figure}
%%%%%%%%%%%%%%%%%%%%%%%%%%%%%%%%%%%%%%%%%%%%%%%%%%%%%%%%%%%%%%%%%%%%%%%%%%%%%%%%%%%%
\section{Results}
\label{res}
To verify if the implemented control structure described in the previous chapter works properly on the real system, several input trajectory signals were defined, which the system was supposed to follow. 
\\In the following figure \ref{fig:step}, the system is supposed to follow step signals. 
\begin{figure}[!h]
\centering
    \begin{subfigure}{.5\textwidth}
      \centering
      \includesvg[width=1\linewidth]{img/01.svg}
      \caption{Step-reference 1}
      \label{fig:01}
    \end{subfigure}%
    \begin{subfigure}{.5\textwidth}
      \centering
      \includesvg[width=1\linewidth]{img/02.svg}
      \caption{Step-reference 2}
      \label{fig:02}
    \end{subfigure}
    \caption{Response to step trajectory references}
    \label{fig:step}
\end{figure}

As it can be seen, the hardware follows the trajectory with a certain overshoot . Additionally, differently from the simulation, the real hardware oscillates quite a bit around the desired step-setpoint.   case. In this figure, the voltage represents to the distance measured by the sensor. Through the sensor characteristic, this voltage can be converted to a distance value. The voltage to ball-height conversion can be seen in the following figure \ref{fig:volt_dist}. 
\begin{figure}[!h]
        \centering
        \includesvg[width=0.5\linewidth]{polyfit_sens_data.svg}
        \caption[Sensor voltage to ball-height characteristic]{Sensor voltage to ball-height characteristic}
        \label{fig:volt_dist}
\end{figure}

It can be seen that sensor does not follow a linear relationship between voltage and height. In addition to this nonlinear sensor characteristic, also the introduced offsets and gains have to be taken into account, when the voltage values are transformed into ball-height values. 
This voltage to height conversion can be applied equally also for the results described subsequently.  
In the following figure \ref{fig:04} the response to a reference trajectory consisting of several steps is shown. 
\begin{figure}[!h]
        \centering
        \includesvg[width=0.5\linewidth]{img/04.svg}
        \caption{Response to a trajectory containing several steps}
        \label{fig:04}
\end{figure}
\\In this figure it can be seen, that the system tries to follow the reference trajectory. However, since the time between the single steps is quite short, the hardware is not able to follow the reference trajectory perfectly. Especially in the first step, the system is not able to follow the reference and has a high overshoot. 
In the following figure \ref{fig:sine} the response to sinusoidal reference inputs is shown.

\begin{figure}[!h]
\centering
    \begin{subfigure}{.5\textwidth}
      \centering
      \includesvg[width=1\linewidth]{img/03.svg}
      \caption{Sine-reference - low freq and low amp}
      \label{fig:06}
    \end{subfigure}%
    \begin{subfigure}{.5\textwidth}
      \centering
      \includesvg[width=1\linewidth]{img/05.svg}
      \caption{Sine-reference 2 - middle freq and high amp}
      \label{fig:06}
    \end{subfigure}
    \begin{subfigure}{.5\textwidth}
      \centering
      \includesvg[width=1\linewidth]{img/06.svg}
      \caption{Sine-reference 3 - high freq and high amp}
      \label{fig:06}
    \end{subfigure}    
    \caption{Response to sinusoidal trajectory references}
    \label{fig:sine}
\end{figure}
As it can be seen, the hardware is able to follow the sinusoidal reference trajectory at low frequencies. However, if the trajectory-frequency is too high, the system response is too slow to follow the trajectory. Additionally, in all three frequency examples it can be seen, that the system has problems to follow the trajectory in the startup. In all cases, this results in a relatively high overshoot.  