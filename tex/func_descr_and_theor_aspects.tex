% !TEX TS-program = pdflatex
% !TEX encoding = UTF-8 Unicode
% !TEX root = ../main.tex
% !TEX spellcheck = en-US
% ****************************************************************************************
% File: func_descr_and_theor_aspects.tex
% Author: Patrick Haselwanter, Christoph Ehrhardt
% Date: 2023-10-28
% ****************************************************************************************
\chapter{Functional Description and Theoretical Aspects}
\label{chapter:func_descr_and_theor_aspects}

\section{System Identification}
\label{sec:sys_ident}
To be able to create a model of the system and validate the controller independently from the real plant, system identification is used. By recording a step response of the real system and 
\begin{itemize}
    \item adjusting it to a format where the step starts at time $t=0$, 
    \item removing the offset of the step response in order that the step starts at $u=0\,$V and
    \item adjusting the gain in order that the step ends at $u=1\,$V, 
\end{itemize}
a mathematical function can be identified, which describes the behaviour of the real system. 
From the following figure \ref{fig:sys_id} it can be seen, that the real system is of second order. 
\begin{figure}[!h]
\centering
    \begin{subfigure}{.5\textwidth}
      \centering
      \includesvg[width=1\linewidth]{img/step_response.svg}
      \caption{Step response of the real system}
      \label{fig:ser}
    \end{subfigure}%
    \begin{subfigure}{.5\textwidth}
      \centering
      \includesvg[width=1\linewidth]{img/fitting.svg}
      \caption{Adjusted and identified system}
      \label{fig:par}
    \end{subfigure}
    \caption{System identification}
    \label{fig:sys_id}
\end{figure}
\\In this figure, also the identified system is already shown. The identified system can be described with the transfer function. 
\begin{equation}
        G(s) =\frac{Y(s)}{U(s)}= \frac{\omega_n^2}{s^2+2 \zeta \omega_n s+\omega_n^2} = \frac{3.74^2}{s^2+2 \cdot 0.60 \cdot3.74 s+3.74^2}
        \label{eqn:tf}
\end{equation}
where G represents the transfer function, $Y(s)$ the output, $U(s)$ the input, $\omega_n$ the natural frequency, $\zeta$ the damping ratio and $s$ the poles. Based on these values, the controller design, which is described in  chapter \ref{chap:contr_des}, can be done. 
In Matlab, the system identification is done with the \textit{lsqcurevefit}-function.
From equation \ref{eqn:tf}, the single values for $K$, the $\omega_n$ and $\zeta$ can be determined. From these values, the state representation of the system can be defined as
\begin{equation}
    \dot{x} = 
    \begin{bmatrix}
        0 & 1 \\
        -\omega_n^2 & -2\zeta\omega_n
    \end{bmatrix}
    x
    +
    \begin{bmatrix}
        0 \\
        1
    \end{bmatrix}
    u = 
     \begin{bmatrix}
        0 & 1 \\
        -3.74^2 & -2\cdot0.6\cdot03.74
    \end{bmatrix}
    x
    +
    \begin{bmatrix}
        0 \\
        1
    \end{bmatrix}
    u
\end{equation}
\begin{equation}
y = K\begin{bmatrix} 1 & 0 \end{bmatrix} x = 1.10\begin{bmatrix} 1 & 0 \end{bmatrix} x
\end{equation}
Additionally, the settling time $t_s$ and the percentage overshoot $p_\%$ can be calculated with 
\begin{equation}
    t_s=\frac{4}{\zeta\omega_n} = \frac{4}{0.6 \cdot 3,74} = 1.79\,\text{s}
\end{equation}
\begin{equation}
    p_\% = 100 \cdot e^{-\frac{\zeta\pi}{\sqrt{1-\zeta^2}}} = 100 \cdot e^{-\frac{0.60\pi}{\sqrt{1-0.60^2}}} = 9.69
\end{equation}

%----------------------------------------------------------------------------------------------------
\section{Controller Design}
\label{chap:contr_des}
In order to define a suitable controller, at first the desired time-characterics of the controlled system have to be defined
\begin{itemize}
\item desired settling time: $t_s = 0.5\,$s 
    \item percentage overshoot: $p_\% = 5\,$\%
\end{itemize}
From that, the desired damping ratio $\zeta$ and the  desired natural frequency can be calculated
\begin{equation}
    \zeta = -\frac{ln p_\%}{\sqrt{\pi^2+ln^2p_\%}} = 0.69
\end{equation}
\begin{equation}
    \omega_n = \frac{4}{\zeta t_s} = 8.89\, \text{s}^{-1}
\end{equation}
\\From that, the desired pole locations can be calculated with
\begin{equation}
    s_{1,2} = -\zeta\omega_n \pm i\omega_n\sqrt{1-\zeta^2} = -6.13\pm6.43
    \label{eqn:poles}
\end{equation}
\vspace{0.5cm}
From this, the following controller gain can be derived. 
\begin{equation}
    \vec{K_C} = [75.27, 7.81]
\end{equation}
The state equation for the servo control is given by 
\begin{equation}
    \left[\begin{array}{c}
    \ddot{\vec{x}} \\
    \dot{e}
    \end{array}\right]=\left[\begin{array}{cc}
    \mathbf{A} & \mathbf{0} \\
    \vec{C} & 0
    \end{array}\right]\left[\begin{array}{l}
    \dot{\vec{x}} \\
    e
    \end{array}\right]+\left[\begin{array}{c}
    \vec{B} \\
    0.
    \end{array}\right] \dot{u}.
\end{equation}
Additionally to the poles derived in \ref{eqn:poles}, a third pole is placed at
\begin{equation}
    s_3 = -1 + 0i.
\end{equation}
After calculating also the prefilter with 
\begin{equation}
    V=-\frac{1}{\vec{C}(\mathbf{A}-\vec{B} \vec{K})^{-1} \vec{B}} = 5.15,
\end{equation}
the servo controller values $K_I$ and $\vec{K_P}$ with
\begin{equation}
    \begin{aligned}
        K_I & =p_I V = -5.14\\
        \vec{K}_P & =\vec{K}+K_I \vec{C}(\boldsymbol{A}-\vec{B} \vec{K})^{-1} = [87.54, 8.81]
    \end{aligned}
\end{equation}
The structure of the calculated state feedback controller is shown  in the following figure \ref{fig:sfc}
\begin{figure}[!h]
    \begin{subfigure}{0.5\textwidth}
        \centering
        \includegraphics[width=1\linewidth]{img/state_feedback_controller.png}
        \caption[State feedback controller ]{State feedback controller\cite{ACE}}
        \label{fig:sfc}
    \end{subfigure}
    \begin{subfigure}{0.5\textwidth}
        \centering
        \includegraphics[width=1\linewidth]{img/state_observer.png}
        \caption[State observer ]{State observer \cite{ACE}}
        \label{fig:obs}
    \end{subfigure}
    \label{fig:sfc_obs}
    \caption{Controller and Observer structure}
\end{figure}

For state feedback control  all state variables of the system are needed. In the context of this aerodynamic levitation system, not all states are measured. Therefore an observer has to be introduced. 
\\The observer is defined based on the equation
\begin{equation}
    \dot{\hat{\vec{x}}}=\mathbf{A} \hat{\vec{x}}+\vec{B} u+\vec{L}(\Delta\vec{y}).
\end{equation}
The observer poles are placed three times further to the left side of the complex plane in order to be three times faster than the poles from \ref{eqn:poles}. This leads to the following observer gain.  
\begin{equation}
    \vec{L} = [2.11; 36.70]
\end{equation}
\\ The structure of the observer is shown in the following figure \ref{fig:obs}. 

%----------------------------------------------------------------------------------------------------
\section{Controller Validation through Simulation}
\label{chap:sim}
To be able to check, whether the designed controller works properly, it is applied to a model of the real system. This plant-model is derived by the system identification described in chapter \ref{sec:sys_ident} and  the controller is implemented based on the concepts described in chapter \ref{chap:contr_des}. The resulting Simulink-Model is shown in the following figure \ref{fig:Simulink_sim_model}.

\begin{figure}[!h]
\centering
    \begin{subfigure}{\textwidth}
      \centering
      \includegraphics[width=1\linewidth]{img/Simulation_layer1.png}
      \caption{Top level servo controlled system}
      \label{fig:simlayer1}
    \end{subfigure}
    \begin{subfigure}{\textwidth}
      \centering
      \includegraphics[width=.8\linewidth]{img/Simulation_layer2.png}
      \caption{Linear system with observer}
      \label{fig:simlayer2}
    \end{subfigure}
    \begin{subfigure}{\textwidth}
      \centering
      \includegraphics[width=.6\linewidth]{img/Simulation_layer3.png}
      \caption{State observer}
      \label{fig:simlayer1}
    \end{subfigure}    
    \caption[Simulink model used to validate the controller]{Simulink model used to validate the controller independently from the real hardware}
    \label{fig:Simulink_sim_model}
\end{figure}
\vspace{0.5cm}

This Simulink model builds also the base for the real hardware implementation described in chapter \ref{sec:HI}. 
To check the behaviour of the system, in the following figure \ref{fig:sim_res} a step signal reference has been set. 
\begin{figure}[!h]
        \centering
        \includesvg[width=10cm]{img/Simulation_Step.svg}
        \caption{Response to a step-reference}
        \label{fig:sim_res}
\end{figure}

As it can be seen, the system is able to follow this step reference. However, it can also be seen that the system is relatively slow. This fact plays also a crucial role in the behaviour of the real system described in chapter \ref{res} 
% EOF